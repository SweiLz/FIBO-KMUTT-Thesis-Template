

\subsection{หุ่นยนต์ฮิวมานอยด์}
หุ่นยนต์ฮิวมานอยด์ คือ หุ่นยนต์ที่ถูกสร้างขึ้นมาให้มีรูปร่างคล้ายคลึงกับสรีระโครงสร้างของมนุษย์
มักถูกออกแบบขึ้นมาเพื่อจุดประสงค์เฉพาะอย่าง เช่น เพื่อให้ใช้เครื่องมือต่างๆของมนุษย์ เพื่อให้อยู่ในสภาพแวดล้อมของมนุษย์
เพื่อศึกษาการเคลื่อนไหนของร่ายกายมนุษย์ เพื่อศึกษาการมองเห็นของมนุษย์ เพื่อทำงานในสิ่งที่มนุษย์ทำได้ยาก
หรือเพื่อวัตถุประสงค์อื่นๆ โดยทั่วไปแล้ว หุ่นยนต์ฮิวมานอยด์จะประกอบไปด้วย 4 ส่วนคือ ส่วนของหัว ส่วนของลำตัว ส่วนของแขน
และส่วนของขา แต่การสร้างหุ่นยนต์ฮิวมานอยด์นั้นก็ไม่จำเป็นที่จะต้องมีส่วนประกอบทุกส่วนดังที่กล่าวไป
ในบางครั้งอาจมีเพียงแค่ส่วนบนเท่านั้น ดังรูปที่ \ref{fig:namo} หุ่นยนต์นะโมจากสถาบันวิทยาการหุ่นยนต์ภาคสนาม
เป็นหุ่นยนต์ที่มีส่วนบนเหมือนมนุษย์ แต่มีส่วนล่างเป็นล้อ หรือหุ่นยนต์ฮิวมานอยด์ที่มีเพียงแค่ส่วนล่าง ดังรูปที่ \ref{fig:ส้มจุก}
หุ่นยนต์ส้มจุก เป็นหุ่นยนต์ฮิวมานอยด์ที่มีเพียงแค่ส่วนขาเท่านั้น หรือหุ่นยนต์ฮิวมานอยด์ที่มีเพียงใบหน้าเหมือนมนุษย์ ดังรูปที่
\ref{fig:โซเฟีย} หุ่นยนต์โซเฟีย เป็นแอนดรอยด์ที่มีหน้าตาคล้ายมนุษย์มาก มีตา มีปาก สามารถพูดปฏิสัมพันธ์กับมนุษย์ได้

\begin{figure}[htbp]
    \centering
    \begin{subfigure}[b]{0.3\textwidth}
        \centering
        \includegraphics[width=\textwidth]{chapter2/images/namo.jpg}
        \caption{หุ่นยนต์ประชาสัมพันธ์นะโม}
        \label{fig:namo}
    \end{subfigure}
    \hfill
    \begin{subfigure}[b]{0.3\textwidth}
        \centering
        \includegraphics[width=\textwidth]{chapter2/images/ส้มจุก.jpg}
        \caption{หุ่นยนต์เดินสองขาส้มจุก}
        \label{fig:ส้มจุก}
    \end{subfigure}
    \hfill
    \begin{subfigure}[b]{0.3\textwidth}
        \centering
        \includegraphics[width=\textwidth]{chapter2/images/โซเฟีย.jpg}
        \caption{หุ่นยนต์แอนดรอยด์โซเฟีย}
        \label{fig:โซเฟีย}
    \end{subfigure}
    \caption{แสดงความแตกต่างของหุ่นยนต์ฮิวมานอยด์แต่ละประเภท}
    \label{fig:diff_humanoid}
\end{figure}

งานวิจัยทางด้านหุ่นยนต์ฮิวมานอยด์จากอดีตจนถึงปัจจุบันส่วนใหญ่จะเป็นการพัฒนาความสามารถของการเดินของหุ่นยนต์
เช่น เริ่มต้นจากแรกสุดจะเป็นการพัฒนาให้หุ่นยนต์สามารถเดินหน้าได้ ต่อมาก็เพิ่มความสามารถให้หุ่นยนต์สามารถเดินบนพื้นเอียง พื้นขรุขระ
เดินเลี้ยวซ้ายขวา เดินขึ้นลงบันได ฯลฯ เป็นต้น นอกจากนี้ยังมีการพัฒนาปรับปรุงสมดุลของการเดินแบบสองขาอีกด้วย สมดุลของการเดินสามารถแบ่งได้สองแบบหลัก
คือ การเดินแบบสมดุลสถิต และการเดินแบบสมดุลพลวัต งานในยุคแรกนั้นจะพัฒนาให้เดินได้แบบสมดุลสถิต ต่อมาเป็นสมดุลกึ่งพลวัต
และเป็นสมดุลพลวัต การพัฒนาตัวควบคุมการเดินของหุ่นยนต์ จำเป็นที่จะต้องใช้ความรู้ทางด้านกลศาสตร์ค่อนข้างมาก มีการใช้สมการที่มีความซับซ้อน

Zheng และคณะ (1988) พัฒนาหุ่นยนต์สองขาที่สามารถเดินบนพื้นราบได้ ให้สามารถเดินต่อเนื่องไปบนพื้นเอียงได้ด้วย
พื้นเอียงที่ใช้มีลักษณะเป็นพื้นเอียงขึ้น หุ่นยนต์ที่ใช้ในงานนี้มีข้อต่อสะโพก (hip), ข้อเท้า (ankle) และลำตัว (torso) มีเซนเซอร์วัดแรงกด (force sensor)
ติดตั้งอยู่ที่ปลายเท้าและส้นเท้าแต่ละข้างเพื่อใช้วัดตำแหน่งของน้ำหนักโดยรวม (center of gravity) ของหุ่นยนต์ การเดินของงานวิจัยจะพิจารณาเฉพาะการเดินในแนวหน้าหลัง
โดยมีหลักการคือ การเดินบนพื้นเอียงโดยที่หุ่นยนต์ยังเดินในท่าทางเหมือนกับตอนที่เดินบนพื้นราบจะทำให้น้ำหนักโดยรวมของหุ่นยนต์เลื่อนไปข้างหลัง
ดังนั้นการที่หุ่นยนต์ขยับลำตัวไปด้านหน้าจำทำให้น้ำหนักโดยรวมของหุ่นยนต์กลับมาอยู่ตรงกลางของพื้นที่รับน้ำหนักเหมือนเดิม
ซึ่งจะทำให้หุ่นยนต์มีความสมดุลได้ ดั้งนั้นข้อมูลที่ได้จากหน่วยวัดแรงกดที่เท้าจะถูกนำมาคำนวณตลอดการเดินเพื่อใช้ในการปรับเปลี่ยนมุมการขยับของลำตัว
การเดินบนพื้นราบเป็นแบบสมดุลสถิตและการเดินบนพื้นเอียงก็ยังคงเป็นแบบสมดุลสถิตเช่นกัน

Inaba และคณะ (1995) สร้างหุ่นยนต์เลียนแบบลิง (ape-like biped) ประกอบด้วยสองมือและสองขา มีการเดินแบบสมดุลสถิต
งานวิจัยนี้มีความคิดว่านอกจากการทำให้หุ่นยนต์สองขาเดินได้โดยไม่ล้มแล้ว ควรจะทำหุ่นยนต์ที่สามารถลุกขึ้นเองได้หลังจากที่ล้มแล้วด้วย
ดังนั้นในงานนี้ หุ่นยนต์ถูกพัฒนาให้สามารถเดิน เมื่อล้มแล้วก็สามารถพลิกตัวและลุกขึ้นมาเดินให้ได้

Kun และ Miller (1996) ได้นำโครงข่ายประสาทเทียม มาประยุกต์ใช้ในการปรับเปลี่ยนท่าทางการเดินโดยอัตโนมัติของหุ่นยนต์สองขา
การที่หุ่นยนต์สามารถปรับเปลี่ยนท่าทางได้โดยอัตโนมัตินี้มีประโยชน์ทำให้หุ่นยนต์เดินได้บนพื้นผิวหลากหลายลักษณะมากขึ้น
ในงานนี้พิจารณาท้ังสมดุลในแนวหน้าหลัง (sigittal plane) และแนวซ้ายขวา (frontal plane) และการเดินของหุุ่นยนต์เป็นแบบสมดุลพลวัต
หลักการทำงานประกอบด้วยตัวสร้างท่าทางการเดินหนึ่งตัว และตัวปรับท่าทางการเดินทั้งแนวหน้าหลังและซ้ายขวาอีกหนึ่งตัว
โดยค่าการปรับเปลี่ยนนั้นจะได้มาจาก แรงกดที่เท้า ความยาวการก้าวเท้า ความสูงของการยกเท้า เป็นต้น นอกจากนี้
ในปีถัดมาทั้งสิงได้ใช้หลักการที่ใช้ในงานนี้ไปใช้กับการเดินของหุ่นยนต์อีกตัว (Kun and Miller, 1997)

Hirai และคณะ (1998) พัฒนาหุ่นยนต์ฮิวมานอยด์ ซึ่งตัวหุ่นยนต์มีความคล้ายมนุษย์มาก สามารถเดินได้อย่างราบลื่นคล้านมนุษย์มากที่สุด
เช่น สามารถเดินได้ในพื้นผิวชนิดต่างๆ เดินได้บนพื้นเอียงขึ้นเอียงลง เดินขึ้นลงบันได้ได้ เดินเข็นรถได้ เป็นต้น การเดินในทุกสถานการณ์เป็นการเดินแบบสมดุลพลวัต
หุ่นยนต์สามารถเดินได้ด้วยความเร็วสูงสุด 4.7 กิโลเมตรต่อชั่วโมง หุ่นยนต์ประกอบไปด้วย แขนข้างละ 9 องศาอิสระ ขาข้างละ 6 องศาอิสระ
ที่บริเวณหัวมีกล้องติดตั้งอยู่ 4 ตัว นอกจากนี้ยังมีอุปกรณ์ที่ใช้ในการรักษาสมดุลอื่นๆ อีกได้แก่ IMU ที่ติดตั้งบริเวณลำตัว และ Force sensor ที่ติดที่เท้าทั้งสองข้าง

ส่วนประกอบของหุ่นยนต์ฮิวมานอยด์สามารถจำแนกออกเป็นส่วนหลักๆได้สามส่วนคือ 
ส่วนการรับรู้ ส่วนการประมวลผล และส่วนการขับเคลื่อน

\subsubsection*{การรับรู้ของหุ่นยนต์ฮิวมานอยด์}
Sensing on humanoid robots is more difficult than on other types of robots since the
movement of the robot leads to noise in the sensors.  For example, camera images
get blurry if the shutter speed is too slow or if it is a rolling shutter sensor.  Also,
odometry  data  is  less  certain  than  on  wheeled  robots.   Furthermore,  the  position
of sensors in relation to the world is not stable.  On a wheeled robot, the camera
is  usually  a  fixed  hight  above  the  ground.   A  humanoid  robot  has  to  do  forward
kinematics from its support feet to the camera to get the position and orientation.
This also requires knowing which leg is the supporting leg and if the robot is standing
at all. For this an inertia measurement unit (IMU) and position encoders in the joints
are crucial.  These are described in the following.
The  IMU  is  a  combination  of  a  gyroscope  and  an  accelerometer.   It  measures
angular velocities and linear accelerations.  It is normally installed inside the torso
near the center of mass.  The sensor values can be used to detect falls and to identify
the side on which the robot landed after a fall.
Furthermore,  the  positions  of  the  robot’s  joints  have  to  be  measured.   This  is
usually  realized  by  a  combination  of  a  magnet  on  the  end  of  the  outermost  gear
and a hall sensor [Ramsden, 2011].  When the gear rotates, the magnet field rotates
accordingly.  The hall sensor measures this and a microcontroller can compute the
position of the servo based on this data.
\subsubsection*{การประมวลผลของหุ่นยนต์ฮิวมานอยด์}
In  the  early  days  of  humanoid  robots,  having  enough  computational  power  was
difficult because the  space and  mass for  the computational  unit are  very  limited.
Today, single-board computers, e.g.  Raspberry Pi [Upton and Halfacree, 2014], and
Intel NUCs [Perico et al., 2014] are mostly used, depending on the size of the robot.
All these boards provide GPUs and multiple CPU cores.  Therefore the interest in
parallelizing software and outsourcing computation to the GPU, especially for neural
networks  and  computer  vision,  rose  in  the  last  years.   Sometimes  multiple  cheap
single-board computers are installed and connected using an Ethernet network, to
get  a  high  performance  for  low  cost.   This  comes  with  the  disadvantage  that  the
software has to be able to run in parallel on a distributed system.
\subsubsection*{การขับเคลื่อนของหุ่นยนต์ฮิวมานอยด์}
A humanoid robot is typically composed of multiple joints.  Each joint has to be
actuated which can be done in two different ways.  Either the joint can be moved
by tendons, similar to the human movement, or a motor can be placed directly into
the  joint  axis.   The  later  approach  is  used  more  often  in  smaller  robots,  since  it
needs less space.  Using tendons makes exact positioning of the joint more difficult,
but also enables to use bigger and linear motors, since they don’t have to be placed
inside the joint.
Actuators need a high strength in relation to their weight, since they have to carry
themselves.  Multiple actuators are typically needed for one limb, resulting in a high
number of cables, which have to be laid over multiple joints.  Therefore a bus system
is often used to limit the number of cables.  The bus requires the motors to have a
micro controller, to receive goal positions and to send the current position.  Micro
controller, motor and a gearbox are usually grouped together in a case and called
servo
.  For reaching the goal position and for holding it, a controller is required in
the firmware of the servo.  This is usually a PID controller [Wescott, 2000].


\clearpage
\subsection{ทฤษฏีที่เกี่ยวข้องกับมนุษย์}
\subsubsection{การวิเคราะห์การเดินของมนุษย์}
\hspace*{10mm} การเคลื่อนที่ของหุ่นยนต์ฮิวมานอยด์นั้นจะเลียนแบบมาจากการเดินของมนุษย์
ดั้งนั้นการวิเคราะห์ลักษณะการเดินของมนุษย์ จะเป็นการศึกษาเพื่อทำความเข้าใจถึงธรรมชาติการเดิน
ก่อนนำไปทำการออกแบบกลไกทางกลและระบบควบคุมของหุ่นยนต์ฮิวมานอยด์
การก้าวเดินของมนุษย์โดยปกติแล้ว จะมีลักษณะเป็นวัฏจักร วนซ้ำไปเรื่อยๆ ในทิศทางที่ต้องการจนกว่าจะทำการหยุดเดิน
การทรงตัวในระหว่างการยืนหรือการเดินนั้น เป็นไปตามสัญชาติญาณซึ่งเกิดจากการรักษาความสมดุลของระดับน้ำในหู\footnote{text}
ส่งสัญญาณผ่านเส้นประสาทไปยังกล้ามเนื้อส่วนต่างๆ ที่ทำหน้าที่ให้เกิดการเคลื่อนที่ \\
\hspace*{10mm} การเคลื่อนที่ของมนุษย์ในการเดินไปข้างหน้าสามารถแบ่งออกเป็นช่วงต่างๆดังนี้
\\
\begin{figure}[htbp]
    \centering
    \includegraphics[width=0.55\textwidth]{chapter2/images/gaitcycle.png}
    \caption{วัฏจักรการเดินของมนุษย์}
    \label{fig:human_gait_cycle}
\end{figure}
\begin{enumerate}[label=\arabic*., leftmargin=1.5cm]
    \item ช่วงเริ่มการวางเท้าเพื่อเข้าสู่ช่วงเริ่มต้นเหวี่ยงเท้า เป็นช่วงที่เท้าเกิดการกระแทกลงบนพื้นหลังจากทำการเหวี่ยงมาจากด้านหลัง
    โดยธรรมชาติมนุษย์จะทำการวางส้นเท้าลงเพื่อลดแรงกระแทกที่เกิดขึ้นในช่วงนี้
    ดังนั้นทางกายภาพในส่วนของส้นเท้ามนุษย์จึงมีลักษณะอ่อนนุ่ม
    \item ช่วงเริ่มต้นเหวี่ยงเท้าเพื่อเข้าสู่ช่วงเหวี่ยงเท้า หลังจากทำการวางส้นเท้าลงกับพื้นแล้ว ข้อเข้าจะปรับมุมเพื่อให้ฝ่าเท้าแนบพื้นสนิท
    ขณะเดียวกันขาอีกข้างจะยกสูงขึ้นเพื่อถ่ายเทน้ำหนักไปยังเท้าที่เพิ่งวางลง
	\item ช่วงเหวี่ยงเท้า เป็นช่วงที่ขาหนึ่งยกลอยอยู่ในอากาศและขาที่วางแนบกับพื้นจะรองรับน้ำหนักทั้งหมดของร่างกาย
	\item ช่วงเตรียมการวางเท้า เป็นช่วงที่ขาข้างที่ลอยอยู่เหวี่ยงไปข้างหน้าเพื่อเตรียมเข้าสู่ช่วงรองรับ 
    ในขณะเดียวกันขาที่รับน้ำหนักอยู่จะทำการผลักตัวเพื่อเริ่มทำการถ่ายเทน้ำหนักไปข้างหน้า
\end{enumerate}

\subsubsection{การวิเคราะห์องศาอิสระของมนุษย์}

\subsubsection{กายวิภาคศาสตร์}

\subsection{ทฤษฏีที่เกี่ยวข้องกับหุ่นยนต์ฮิวมานอยด์}
\subsubsection{ส่วนประกอบของหุ่นยนต์ฮิวมานอยด์}
\begin{figure}[ht]
	\centering
	\includegraphics[width=0.4\textwidth]{chapter2/images/robot_component.png}
	\caption{ส่วนประกอบของหุ่นยนต์ฮิวมานอยด์}
	\label{fig:robot_component}
\end{figure}
หุ่นยนต์ฮิวมานอยด์ประกอบด้วยก้านต่อหลายๆก้านที่นำมาต่อกัน ลักษณะโครงสร้างนั้นจะเป็นแบบโซ่เปิด (Open kinematic chain)
และแต่ละก้านต่อจะเชื่อมต่อกันด้วยข้อต่อแบบหมุน เราสามารถแบ่งโครงสร้างของหุ่นยนต์ฮิวมานอยด์ออกเป็นส่วนหลักๆเป็น 2 ส่วน ส่วนแรกคือ
ส่วนก้านต่อของลำตัวหุ่นยนต์ (Torso) ซึ่งเราสามารถที่จะรวมไปถึงส่วนแขนกับหัวด้วย
และในส่วนที่สองคือ ส่วนก้านต่อของขาหุ่นยนต์ (Legs) ซึ่งเป็นส่วนขาของหุ่นยนต์ทั้งสองข้างที่สามารถนำไปที่สัมผัสกับพื้นได้
ทั้งสองก้านต่อนี้ถูกเชื่อมต่อกันด้วยส่วนของสะโพก (Hip) ที่อยู่ระหว่างส่วนลำตัวกับส่วนของขาหุ่นยนต์ ดังรูปที่ \ref{fig:robot_component}

\subsubsection{วัฏจักรการเดินของหุ่นยนต์ฮิวมานอยด์}
วัฏจักรการเดินของหุ่นยนต์ คือ การที่หุ่นยนต์จะต้องมีการถ่ายน้ำหนักไปมาระหว่างเท้าซ้ายและเท้าขวา
มีบางช่วงที่น้ำหนักตกลงบนเท้าข้างใดข้างหนึ่งหรือทั้งสองข้างพร้อมกัน สามารถแบ่งออกเป็นช่วงได้สองช่วง คือช่วงการยืนด้วยขาข้างเดียว
และช่วงการยืนด้วยขาทั้งสองข้าง
\begin{figure}[!ht]
	\centering
	\begin{subfigure}[b]{0.22\textwidth}
		\centering
		\includegraphics[width=\textwidth]{chapter2/images/doublesupport.png}
		\caption{ยืนด้วยขาสองข้าง}
		\label{fig:robot_walk_1}
	\end{subfigure}
	\hfill
	\begin{subfigure}[b]{0.45\textwidth}
		\centering
		\includegraphics[width=\textwidth]{chapter2/images/singlesupport.png}
		\caption{ยืนด้วยขาข้างเดียว}
		\label{fig:robot_walk_2}
	\end{subfigure}
	\hfill
	\begin{subfigure}[b]{0.22\textwidth}
		\centering
		\includegraphics[width=\textwidth]{chapter2/images/doublesupport2.png}
		\caption{ยืนด้วยขาสองข้าง}
		\label{fig:robot_walk_3}
	\end{subfigure}
	\caption{วัฐจักรการเดินของหุ่นยนต์ฮิวมานอยด์}
	\label{fig:robot_walk_phase}
\end{figure}

\clearpage
\paragraph*{1) การยืนด้วยขาข้างเดียว :}
เป็นช่วงที่มีเท้าของหุ่นยนต์สัมผัสพื้นเพียงข้างเดียว ส่วนเท้าอีกข้างของหุ่นยนต์จะถูกยกลอยจากพื้น
โดยที่ไม่มีส่วนใดๆของขาข้างนั้นสัมผัสกับพื้นเลย ช่วงนี้จะเกิดขึ้นเมื่อมีการแกว่งเท้าจากข้างหลังไปข้างหน้า
ดังรูปที่ \ref{fig:robot_walk_2}

\paragraph*{2) การยืนด้วยขาสองข้าง :}
เป็นช่วงที่เท้าทั้งสองข้างของหุ่นยนต์สัมผัสกับพื้น ช่วงนี้จะเกิดตั้งแต่หุ่นยนต์วางเท้าขณะที่ส้นเท้าแตะกับพื้น
ไปจนถึง ปลายเท้าของขาอีกข้างหลุดออกจากพื้น

การเดินได้โดยไม่ล้มนั้น ตัวหุ่นยนต์จะต้องรักษาสมดุลของการเดินให้ได้ตลอดช่วงเวลาของการเดิน
ซึ่งสมดุลของการเดินแบบสองขาสามารถแบ่งตามลักษณะการเดินและการถ่ายน้ำหนักได้เป็น 2 รูปแบบหลัก คือ 
การเดินแบบสมดุลสถิต (static balance walking) และ การเดินแบบสมดุลพลวัต (dynamic balance walking)

\subsubsection{การสร้างและการควบคุมการเดินแบบสมดุลสถิต}
การเดินของหุ่นยนต์ในลักษณะนี้ จุดศูนย์กลางมวล (CoM) ของตัวหุ่นยนต์จะไม่มีการเคลื่อนไหวออกนอกบริเวณฐานรับน้ำหนัก (Supporting Area)
ตลอดช่วงเวลาการเดิน ไม่ว่าจะเป็นช่วงเวลาที่รับน้ำหนักด้วยเท้าข้างเดียวหรือทั้งสองข้างก็ตาม หมายความว่า โครงสร้างของหุ่นยนต์จะไม่ล้มแน่นอน
เนื่องจากการสร้างรูปแบบการเดินด้วยวิธีนี้จะควบคุมให้ตำแหน่งของจุดศูนย์กลางมวล อยู่ภายในพื้นที่ฐานรับน้ำหนักของหุ่นยนต์ตลอดเวลา
\ref{Legged robots walk the walk,https://blog.csiro.au/legged-robots-walk-walk/}

\begin{figure}[ht]
	\centering
	\includegraphics[width=0.4\textwidth]{chapter2/images/cominsupportpolygon.png}
	\caption{การควบคุมตำแหน่งของจุดรวมมวลให้อยู่ในพื้นที่ฐาน}
	\label{fig:robot_com_support}
\end{figure}

ข้อดีของการสร้างและควบคุมการเดินของหุ่นยนต์ด้วยวิธีนี้คือ สามารถสร้างรูปแบบการเดินได้โดยที่มีความซับซ้อนไม่มากนัก
สามารถสั่งให้หุ่นยนต์หยุดค้างในท่าทางใดๆก็ได้ตลอดเวลาโดยหุ่นยนต์ไม่ล้ม หุ่นยนต์ที่มีฝ่าเท้าใหญ่จะทำให้ง่ายต่อการก้าวเดินมากขึ้น
นอกจากการควบคุมการก้าวขาแล้วอาจเพิ่มการควบคุมส่วนลำตัวเพิ่มเติม เพื่อเป็นการเพิ่มเสถียรภาพในการเดินและการถ่ายเทน้ำหนัก
โดยที่อาจจะมีการเพิ่มเซนเซอร์วัดแรงที่ฝ่าเท้าเพื่อตรวจสอบการกระจายแรงกดที่ฝ่าเท้า เพื่อตรวจสอบว่าตำแหน่งของจุดรวมน้ำหนักอยู่บนพื้นที่ฝ่าเท้าหรือไม่
หรือเพื่อตรวจสอบเสถียรภาพของการเดินเพื่อแก้ไขท่าทางการเดินไม่ให้เกิดการล้ม

ข้อเสียของการควบคุมการเดินด้วยวิธีนี้คือ หุ่นยนต์จะใช้เวลาในการก้าวเดินมาก ใช้พลังงานในการเดินมากกว่าการเดินแบบสมดุลพลวัต
และท่าทางที่ได้จะมีความแตกต่างจากท่าทางการเดินของมนุษย์

\subsubsection{การสร้างและการควบคุมการเดินแบบสมดุลพลวัต}
การสร้างรูปแบบการเดินและควบคุมการเดินในลักษณะนี้ท่าทางการเดินของหุ่นยนต์นั้นจะคล้ายกับการเดินของมนุษย์มากกว่าแบบสถิต
เนื่องจากมีหลักการในการสร้างท่าทางที่เหมือนกับการเดินของมนุษย์ซึ่งมีขั้นตอนดังนี้คือ เอียงตัวให้ล้มไปในทิศทางที่ต้องการเดิน
เมื่อเริ่มเกิดการล้มขึ้นหุ่นยนต์จะเปลี่ยนตำแหน่งการวางเท้าไปยังตำแหน่งใหม่ เพื่อปรับให้โครงสร้างเข้าสู่สภาวะสมดุลอีกครั้ง

โดยธรรมชาติแล้วมนุษย์มีการถ่ายน้ำหนักในขณะที่เคลื่อนที่หรือยืนอยู่กับที่เพื่อรักษาสมดุลของท่าทางนั้นไว้
แต่หากการถ่ายโอนน้ำหนักนั้นเกิดสภาวะไม่สมดุล ร่างกายจะปรับสภาพโดยการเคลื่อนตำแหน่งของเท้าซึ่งเป็นพื้นที่ฐานออกจากเดิมไปยังตำแหน่งใหม่
เพื่อรักษาสมดุลไว้ หลักการดังกล่าวถูกนำมาใช้กับการควบคุมการเดินของหุ่นยนต์ฮิวมานอยด์ ในขณะที่หุ่นยนต์กำลังเคลื่อนไหว
ผลจากแรงเฉื่อยของการเคลื่อนที่และผลจากแรงดึงดูดของโลกมีผลต่อการเพิ่มและลดความเร่งให้การเดินของหุ่นยนต์
แรงเหล่านี้เรียกว่าแรงเฉื่อยรวมของการเคลื่อนที่ และเมื่อเท้าหุ่นยนต์สัมผัสกับพื้นจะได้รับผลกระทบของแรงนี้ เรียกว่า
แรงปฏิกิริยาจากพื้น

การตัดกันระหว่างแรงปฏิกิริยาจากพื้นและแนวแรงเฉื่อยรวม ตำแหน่งนั้นหากทำให้โมเมนต์เท่ากับศูนย์
เรียกจุดตัดนี้ว่าจุดโมเมนต์ศูนย์ ($ZMP_{robot}$) และจุดที่แรงปฏิกิริยาลงสู่พื้นว่า จุดปฏิกิริยาพื้นฐาน 
ท่าทางการเดินของหุ่นยนต์จะถูกกำหนดและถูกส่งให้กับชุดควบคุมข้อต่อจุดต่างๆของหุ่นยนต์ โดยให้สอดคล้องกับแรงเฉื่อยรวมที่เกิดขึ้นจากการคำนวณ
เรียกว่าแรงเฉื่อยรวมเป้าหมาย และจุดโมเมนต์ศูนย์ที่ได้จากการคำนวณเรียกว่าจุดโมเมนต์ศูนย์เป้าหมาย ($ZMP_{target}$)
เมื่อหุ่นยนต์เกิดสมดุลในขณะที่ทำการเดินได้อย่างสมบูรณ์ แนวแกนของแรงเฉื่อยรวมเป้าหมายและแรงปฏิกิริยาที่พื้นจะเป็นตำแหน่งเดียวกัน
แต่ในขณะที่หุ่นยนต์เดินผ่านพื้นผิวที่มีความขรุขระหรือไม่เรียบตำแหน่งสองจุดดังกล่าง จะไม่ใช่ตำแหน่งเดียวกันทำให้หุ่นยนต์เกิดการล้มได้
แรงที่ทำให้เกิดการล้มนี้เกิดจากตำแหน่งของจุดโมเมนต์ศูนย์และตำแหน่งแรงปฏิกิริยารวมที่พื้นไม่ตรงกัน ซึ่งเป็นสาเหตุหลักที่ทำให้เกิดความไม่สมดุลขึ้น
และเมื่อหุ่นยนต์เสียสมดุลระบบที่จะสามารถป้องกันการล้มและทำให้หุ่นยนต์เดินต่อไปได้อย่างต่อเนื่องคือ ระบบควบคุมแรงปฏิกิริยา
ระบบควบคุมจุดโมเมนต์ศูนย์ และระบบควบคุมการวางเท้า\ref{Achieving Stable walking,http://world.honda.com/ASIMO/history/technology2.html}

\begin{figure}[!ht]
	\centering
	\includegraphics[width=0.7\textwidth]{chapter2/images/zmpdynamicwalking.png}
	\caption{การควบคุมตำแหน่งของจุดโมเมนต์ศูนย์ให้ตรงกับแรงปฏิกิริยารวม}
	\label{fig:robot_zmp_support}
\end{figure}

อย่างไรก็ตาม การสร้างท่าทางการเดินในลักษณะนี้ต้องใช้สมการในการคำนวณที่ซับซ้อนมาก
เนื่องจากต้องหาความสัมพันธ์ระหว่างองค์ประกอบหลายส่วน เช่น น้ำหนักของโครงสร้างในแต่ละส่วน 
แรงบิดที่แต่ละข้อต่อ และโมเมนต์โดยรวมของระบบ นอกจากนี้ยังต้องใช้อุปกรณ์การตรวจวัดต่างๆ เช่น เซนเซอร์วัดแรง
เซนเซอร์วัดมุม เซนเซอร์วัดแรงบิด ติดตั้งตามจุดต่างๆของโครงสร้างเพื่อวัดค่าออกมา ก่อนที่จะทำการคำนวณตำแหน่ง
และสร้างท่าทางการเดินของหุ่นยนต์ฮิวมานอยด์ ท่าทางการเดินที่ได้จากการควบคุมด้วยวิธีนี้ จะมีความคล้ายคลึงกับท่าทางการเดินของมนุษย์มาก

\subsubsection{จุดศูนย์กลางมวลของหุ่นยนต์}
หากต้องการให้หุ่นยนต์สามารถที่จะทรงตัวอยู่ได้โดยไม่ล้มนั้น จึงต้องรู้ตำแหน่งจุดศูนย์กลางมวลของหุ่นยนต์ตลอดเวลา
และต้องให้จุดศูนย์กลางมวลฉายตกในบริเวณฐานรับน้ำหนักของหุ่นยนต์โดยหาจากพื้นที่ที่ฝ่าเท้าสัมผัสกับพื้น
วิธีการนี้เป็นวิธีการทางสถิตศาสตร์