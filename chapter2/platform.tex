\subsection{ข้อแตกต่างระหว่าง Open platform กับ Non-open platform}
\hspace*{10mm} หุ่นยนต์ Open platform คือ การออกแบบระบบพื้นฐานของหุ่นยนต์ที่เปิดให้ผู้ที่ต้องการศึกษาหรือผู้ใช้ทั่วไปสามารถเข้าถึงข้อมูลต่างๆที่เกี่ยวข้องกับหุ่นยนต์นั้นๆได้
ผู้ใช้สามารถที่จะนำข้อมูลเหล่านั้นมาแก้ไข ปรับปรุง แต่งเติม หรือเรียนรู้และพัฒนาตามได้ด้วยตนเอง 
ซึ่งข้อมูลที่กล่าวมานั้นสามารถหาได้จากเว็บไซต์ของผู้พัฒนาหุ่นยนต์ ปัจจุบันมีหุ่นยนต์ฮิวมานอยด์ที่เป็นเปิดให้เข้าถึงหลายรูปแบบแตกต่างกันไป \\
\hspace*{10mm} ส่วนหุ่นยนต์ Non-open source platform คือหุ่นยนต์ที่สร้างมาเฉพาะเจาะจงสำหรับการวิจัย การสำรวจ หรือการแข่งขันโดยเฉพาะ
ไม่เปิดให้บุคคลภายนอกเข้าศึกษาหรือแก้ไขปรับปรุง ซึ่งทำให้หุ่นยนต์ประเภทนี้ไม่เหมาะสำหรับผู้วิจัยที่จะเรียนรู้และศึกษาด้วยตนเอง เพราะมีขนาดใหญ่
ใช้ทรัพยากรมาก และการออกแบบมีความซับซ้อน เรียนรู้ยากกว่าหุ่นยนต์แบบ Open platform
