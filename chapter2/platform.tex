\subsection{ความแตกต่างของ Operating Systems}

เป็นที่รู้กันโดยทั่วไปว่าฮาร์ทแวร์และซอฟแวร์ของคอมพิวเตอร์นั้นถูกจัดการโดยโปรแกรมในคอมพิวเตอร์ชื่อว่า
operating system (OS) งานพื้นฐานที่ OS ทำก็เช่น การควบคุมและจองหน่วยความจำ
การจัดลำดับความสำคัญของระบบ คอยดูแลควบคุมอุปกรณ์ต่างๆที่เชื่อมต่อเข้ากับคอมพิวเตอร์
การเชื่อมต่อระบบเน็ตเวิร์ค การจัดการไฟล์ข้อมูล อีกทั้งยังรวมไปถึงการให้บริการต่างๆเช่น
การจัดการกระบวนการประมวลผล จัดการไฟล์ของระบบต่างๆ ระบบป้องกัน อื่นๆ

ปัจจุบันมี OS อยู่หลายตัวเช่น Windows, Mac OS X, UNIX, Solaris BS3000, MS-Dos และอื่นๆ
ซึ่งทั้งหมดนี้เป็นส่วนหนึ่งของระบบคอมพิวเตอร์ที่จะคอยช่วยจัดการและควบคุมดูแลการทำงานต่างๆของคอมพิวเตอร์
ระบบคอมพิวเตอร์นั้นอาจจะอยู่ในรูปแบบอื่นๆเช่น เครื่องที่ใช้ทำงาน, เครื่องเซิฟเวอร์, เครื่องคอมพิวเตอร์ส่วนบุคคล, โทรศัพท์เคลื่อนที่,
อุปกรณ์นำทาง หรือแม้กระทั้งระบบที่มีความฉลาดในตัวมันเองเช่น หุ่นยนต์ และ OS
นั้นจะสามารถทำงานบนฮาร์ทแวร์อุปกรณ์ใดๆก็ได้

\subsection{ข้อแตกต่างระหว่าง Open platform กับ Non-open platform}
หุ่นยนต์ Open platform คือ การออกแบบระบบพื้นฐานของหุ่นยนต์ที่เปิดให้ผู้ที่ต้องการศึกษาหรือผู้ใช้ทั่วไปสามารถเข้าถึงข้อมูลต่างๆที่เกี่ยวข้องกับหุ่นยนต์นั้นๆได้
ผู้ใช้สามารถที่จะนำข้อมูลเหล่านั้นมาแก้ไข ปรับปรุง แต่งเติม หรือเรียนรู้และพัฒนาตามได้ด้วยตนเอง 
ซึ่งข้อมูลที่กล่าวมานั้นสามารถหาได้จากเว็บไซต์ของผู้พัฒนาหุ่นยนต์ ปัจจุบันมีหุ่นยนต์ฮิวมานอยด์ที่เป็นเปิดให้เข้าถึงหลายรูปแบบแตกต่างกันไป

ส่วนหุ่นยนต์ Non-open source platform คือหุ่นยนต์ที่สร้างมาเฉพาะเจาะจงสำหรับการวิจัย การสำรวจ หรือการแข่งขันโดยเฉพาะ
ไม่เปิดให้บุคคลภายนอกเข้าศึกษาหรือแก้ไขปรับปรุง ซึ่งทำให้หุ่นยนต์ประเภทนี้ไม่เหมาะสำหรับผู้วิจัยที่จะเรียนรู้และศึกษาด้วยตนเอง เพราะมีขนาดใหญ่
ใช้ทรัพยากรมาก และการออกแบบมีความซับซ้อน เรียนรู้ยากกว่าหุ่นยนต์แบบ Open platform

\subsection{มาตรฐานหน่วยวัดและการบอกพิกัด}
การใช้หน่วยวัดที่ไม่ตรงกันอาจจะทำให้เกิดปัญหาขึ้นได้ เนื่องจากเป็นแพลตฟอร์มนั้นจะมีบุคคลอื่นช่วยกันพัฒนาหลายคน
จึงควรที่จะมีมาตรฐานในการวัดและการกำหนดพิกัดต่างๆที่ตรงกันเพื่อให้เกิดความชัดเจนในการทำความเข้าใจ

\paragraph*{หน่วยวัด}
การวัดนั้นใช้มาตรฐานการวัดเป็น SI Units ซึ่งมาตรฐานนี้ใช้กันอย่างแพร่หลายและเป็นสากล โดยหน่วยการวัดนี้ได้รับการยืนยันจาก
Bureau International des Poids et Mesures ตามตารางที่ \ref{tab:standart_unit}

\begin{table}[!ht]
	\centering
	\begin{tabular}{| l | l | l |}
		\hline
		ปริมาณ (Quantity)                     & หน่วยวัด (Unit)                      & สัญลักษณ์ (Symbol) \\
		\hline
		ความยาว (Length)                    & เมตร (metre)                                 & $m$                                 \\
		มวล (Mass)                                  & กิโลกรัม (kilogram)                  & $kg$                                \\
		เวลา (Time)                               & วินาที (second)                          & $s$                                 \\
		กระแสไฟฟ้า (Electric Current) & แอมแปร์ (ampere)                       & $A$                                 \\
		มุม (Angle)                                 & เรเดียน (radian), องศา (degree) & $rad, deg$                          \\
		ความถี่ (Frequency)                 & เฮิร์ท (Hertz)                           & $Hz$                                \\
		แรง (Force)                                 & นิวตัน (Newton)                          & $N$                                 \\
		กำลัง (Power)                           & วัตต์ (Watt)                               & $W$                                 \\
		แรงดันไฟฟ้า (Voltage)       & โวลต์ (Volt)                               & $V$                                 \\
		อุณหภูมิ (Temperature)            & เซลเซียส (Celsius)                   & $°C$                                 \\
		\hline
	\end{tabular}
	\caption{ตารางแสดงหน่วยวัดมาตฐาน}
	\label{tab:standart_unit}
\end{table}

\paragraph*{พิกัดเฟรม}
การบอกทิศทางการหมุนนั้นใช้หลักตามกฏมือขวา โดยการตั้งแกนนั้นหากเทียบกับมือแล้ว X ไปข้างหน้า Y ไปทางซ้าย Z พุ่งขึ้น ดังรูปที่ \ref{fig:right_hand_rule}
\begin{figure}[!ht]
	\centering
	\includegraphics[width=0.15\textwidth]{chapter2/images/right_hand_rule.png}
	\caption{การตั้งแกนตามกฏมือขวา}
	\label{fig:right_hand_rule}
\end{figure}

\subsection{Robot Operating System}
ROS เป็นกรอบการทำงานที่ได้รับความนิยมและมีประสิทธิภาพในการทำงานมากที่สุดในปัจจุบัน เนื่องจาก ROS
ได้รวบรวมซอฟต์แวร์เครื่องมือที่หลากหลายเอาไว้เป็นหมวดหมู่ เช่น การเชื่อมต่อกับฮาร์ดแวร์
การสร้างระบบควบคุมให้กับอุปกรณ์ต่างๆ อีกทั้งสามารถที่จะเขียนโปรแกรมแล้วนำกลับมาใช้ใหม่ได้
ภายในระบบมีกระบวนการรับส่งข้อมูลต่างๆเป็นของตัวเอง ทำให้ช่วยลดความซับซ้อนและเพิ่มประสิทธิภาพในการทำงานกับแพลตฟอร์มของหุ่นยนต์
กระบวนการเขียนโปรแกรมของ ROS นั้นจะใช้รูปแบบ Graph architecture ซึ่งจะทำให้สามารถแบ่งโปรแกรมต่างๆออกเป็นส่วนๆ เช่น
เซนเซอร์หลายๆตัว ระบบควบคุม ระบบวางแผน ระบบขับเคลื่อน ระบบสื่อสารภายนอก ด้วยตัวระบบของ ROS นั้น ไม่ใช้ Real Time OS
แต่อย่างไรก็ตามเราสามารถใช้งานผสมกับ Real Time ได้

ROS ประกอบไปด้วยแพ็กเกจต่างๆ มาประกอบกันเป็น Node โดยมีตัวกลางทำหน้าที่ในการติดต่อสื่อสารระหว่างอุปกรณ์ที่เป็น Node ต่างๆ
ให้สามารถส่งข้อมูลหากันได้ รูปแบบการสื่อสารใน ROS จะใช้หลักการแบบ Publish/Subscribe ทำให้ไม่จำเป็นที่จะต้องระบุโปรแกรมที่จะรับ
ภาษาในการพัฒนามีให้เลือกที่หลากหลาย เช่น C++, Python, Lisp, MATLAB หรือ JavaScript 

\subsubsection*{ประโยชน์จากการใช้ ROS}
ROS เป็นกรอบการทำงาน ที่อยู่ระหว่าง OS และ Robot ทำให้เราไม่ต้องกังวลเรื่องการจัดการระบบภายในเพราะ
ROS จะช่วยจัดการให้เราทั้งหมด ก่อนจะมี ROS นั้น นักวิจัยจะต้องใช้เวลาไปกับพัฒนาพื้นฐานให้หุ่นยนต์
ซึ่งจะต้องมีทักษะทางด้านเครื่องกล ไฟฟ้า และโปรแกรม ซึ่งบ่อยครั้งที่นักวิจัยหรือนักพัฒนานั้นไม่มีความรู้
หรือประสบการณ์ในการสร้างหุ่นยนต์ ทำให้การทำงานเป็นไปด้วยความลำบาก