\subsubsection{การวิเคราะห์การเดินของมนุษย์}
การเคลื่อนที่ของหุ่นยนต์ฮิวมานอยด์นั้นจะเลียนแบบมาจากการเดินของมนุษย์
ดั้งนั้นการวิเคราะห์ลักษณะการเดินของมนุษย์ จะเป็นการศึกษาเพื่อทำความเข้าใจถึงธรรมชาติการเดิน
ก่อนนำไปทำการออกแบบกลไกทางกลและระบบควบคุมของหุ่นยนต์ฮิวมานอยด์
การก้าวเดินของมนุษย์โดยปกติแล้ว จะมีลักษณะเป็นวัฏจักร วนซ้ำไปเรื่อยๆ ในทิศทางที่ต้องการจนกว่าจะทำการหยุดเดิน
การทรงตัวในระหว่างการยืนหรือการเดินนั้น เป็นไปตามสัญชาติญาณซึ่งเกิดจากการรักษาความสมดุลของระดับน้ำในหู\footnote{text}
ส่งสัญญาณผ่านเส้นประสาทไปยังกล้ามเนื้อส่วนต่างๆ ที่ทำหน้าที่ให้เกิดการเคลื่อนที่

การเคลื่อนที่ของมนุษย์ในการเดินไปข้างหน้าสามารถแบ่งออกเป็นช่วงต่างๆดังนี้
\begin{figure}[htbp]
    \centering
    \includegraphics[width=0.55\textwidth]{chapter2/images/gaitcycle.png}
    \caption{วัฏจักรการเดินของมนุษย์}
    \label{fig:human_gait_cycle}
\end{figure}
\begin{enumerate}[label=\arabic*., leftmargin=1.5cm]
    \item ช่วงเริ่มการวางเท้าเพื่อเข้าสู่ช่วงเริ่มต้นเหวี่ยงเท้า เป็นช่วงที่เท้าเกิดการกระแทกลงบนพื้นหลังจากทำการเหวี่ยงมาจากด้านหลัง
    โดยธรรมชาติมนุษย์จะทำการวางส้นเท้าลงเพื่อลดแรงกระแทกที่เกิดขึ้นในช่วงนี้
    ดังนั้นทางกายภาพในส่วนของส้นเท้ามนุษย์จึงมีลักษณะอ่อนนุ่ม
    \item ช่วงเริ่มต้นเหวี่ยงเท้าเพื่อเข้าสู่ช่วงเหวี่ยงเท้า หลังจากทำการวางส้นเท้าลงกับพื้นแล้ว ข้อเข้าจะปรับมุมเพื่อให้ฝ่าเท้าแนบพื้นสนิท
    ขณะเดียวกันขาอีกข้างจะยกสูงขึ้นเพื่อถ่ายเทน้ำหนักไปยังเท้าที่เพิ่งวางลง
	\item ช่วงเหวี่ยงเท้า เป็นช่วงที่ขาหนึ่งยกลอยอยู่ในอากาศและขาที่วางแนบกับพื้นจะรองรับน้ำหนักทั้งหมดของร่างกาย
	\item ช่วงเตรียมการวางเท้า เป็นช่วงที่ขาข้างที่ลอยอยู่เหวี่ยงไปข้างหน้าเพื่อเตรียมเข้าสู่ช่วงรองรับ 
    ในขณะเดียวกันขาที่รับน้ำหนักอยู่จะทำการผลักตัวเพื่อเริ่มทำการถ่ายเทน้ำหนักไปข้างหน้า
\end{enumerate}

\subsubsection{การวิเคราะห์องศาอิสระของมนุษย์}
การที่มนุษย์เราสามารถเคลื่อนที่ได้นั้น เป็นผลเนื่องมาจากการเคลื่อนที่ของข้อต่อต่าง ๆ ที่อยู่
บนขา ซึ่งประกอบไปด้วย ข้อต่อส่วนสะโพก ข้อต่อส่วนหัวเข่า และข้อต่อส่วนข้อเท้า แรงบิดที่เกิด
ขึ้นของแต่ละข้อต่อมีความสัมพันธ์ต่อกัน ส่งผลให้เกิดเสถียรภาพในการเดินของมนุษย์ เมื่อวิเคราะห์
ลักษณะโครงสร้างในแต่ละส่วน พบว่าข้อต่อส่วนสะโพกมีลักษณะเป็นทรงกลม ทำให้ข้อต่อส่วน
สะโพกสามารถหมุนได้ 3 องศาอิสระ ส่วนหัวเข่าของมนุษย์ มีจุดต่อของข้อที่มีลักษณะเป็นทรงกลม
สองลูกประกอบเข้าด้วยกันทำให้การเคลื่อนที่ถูกบังคับให้สามารถเคลื่อนที่ได้เพียง 1 องศาอิสระ ใน
ส่วนของข้อเท้ามีลักษณะการเคลื่อนที่เหมือนสะโพกคือสามารถเคลื่อนที่ได้ 3 องศาอิสระ

จากทั้งหมดที่ได้ทำการวิเคราะห์มาข้างต้นพบว่าในขาหนึ่งข้างของมนุษย์ประกอบด้วย 7 องศาอิสระ
ซึ่งส่งผลให้การเคลื่อนที่ของมนุษย์มีความคล่องแคล่วสูง แต่ในทางออกแบบกลไกการเดินและการควบคุม
ของหุ่นยนต์สองขาถือว่ามีจำนวนองศาอิสระเกินความจำเป็นในการเคลื่อนที่บนปริภูมิ(space) และยากต่อ
การควบคุม(under actuated) ดังนั้นการกำหนดจำนวนองศาอิสระเพื่อให้หุ่นยนต์เดินได้เสมือนมนุษย์จึง
มีผลในการออกแบบกลไกทางกลและการควบคุมของหุ่นยนต์สองขา 

\subsubsection{กายวิภาคศาสตร์}
