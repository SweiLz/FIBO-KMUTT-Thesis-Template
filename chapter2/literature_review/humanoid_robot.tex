หุ่นยนต์ฮิวมานอยด์ คือ หุ่นยนต์ที่ถูกสร้างขึ้นมาให้มีรูปร่างคล้ายคลึงกับสรีระโครงสร้างของมนุษย์
มักถูกออกแบบขึ้นมาเพื่อจุดประสงค์เฉพาะอย่าง เช่น เพื่อให้ใช้เครื่องมือต่างๆของมนุษย์ เพื่อให้อยู่ในสภาพแวดล้อมของมนุษย์
เพื่อศึกษาการเคลื่อนไหนของร่ายกายมนุษย์ เพื่อศึกษาการมองเห็นของมนุษย์ เพื่อทำงานในสิ่งที่มนุษย์ทำได้ยาก
หรือเพื่อวัตถุประสงค์อื่นๆ โดยทั่วไปแล้ว หุ่นยนต์ฮิวมานอยด์จะประกอบไปด้วย 4 ส่วนคือ ส่วนของหัว ส่วนของลำตัว ส่วนของแขน
และส่วนของขา แต่การสร้างหุ่นยนต์ฮิวมานอยด์นั้นก็ไม่จำเป็นที่จะต้องมีส่วนประกอบทุกส่วนดังที่กล่าวไป
ในบางครั้งอาจมีเพียงแค่ส่วนบนเท่านั้น ดังรูปที่ \ref{fig:namo} หุ่นยนต์นะโมจากสถาบันวิทยาการหุ่นยนต์ภาคสนาม
เป็นหุ่นยนต์ที่มีส่วนบนเหมือนมนุษย์ แต่มีส่วนล่างเป็นล้อ หรือหุ่นยนต์ฮิวมานอยด์ที่มีเพียงแค่ส่วนล่าง ดังรูปที่ \ref{fig:ส้มจุก}
หุ่นยนต์ส้มจุก เป็นหุ่นยนต์ฮิวมานอยด์ที่มีเพียงแค่ส่วนขาเท่านั้น หรือหุ่นยนต์ฮิวมานอยด์ที่มีเพียงใบหน้าเหมือนมนุษย์ ดังรูปที่
\ref{fig:โซเฟีย} หุ่นยนต์โซเฟีย เป็นแอนดรอยด์ที่มีหน้าตาคล้ายมนุษย์มาก มีตา มีปาก สามารถพูดปฏิสัมพันธ์กับมนุษย์ได้

\begin{figure}[htbp]
    \centering
    \begin{subfigure}[b]{0.3\textwidth}
        \centering
        \includegraphics[width=\textwidth]{chapter2/images/namo.jpg}
        \caption{หุ่นยนต์ประชาสัมพันธ์นะโม}
        \label{fig:namo}
    \end{subfigure}
    \hfill
    \begin{subfigure}[b]{0.3\textwidth}
        \centering
        \includegraphics[width=\textwidth]{chapter2/images/ส้มจุก.jpg}
        \caption{หุ่นยนต์เดินสองขาส้มจุก}
        \label{fig:ส้มจุก}
    \end{subfigure}
    \hfill
    \begin{subfigure}[b]{0.3\textwidth}
        \centering
        \includegraphics[width=\textwidth]{chapter2/images/โซเฟีย.jpg}
        \caption{หุ่นยนต์แอนดรอยด์โซเฟีย}
        \label{fig:โซเฟีย}
    \end{subfigure}
    \caption{แสดงความแตกต่างของหุ่นยนต์ฮิวมานอยด์แต่ละประเภท}
    \label{fig:diff_humanoid}
\end{figure}

งานวิจัยทางด้านหุ่นยนต์ฮิวมานอยด์จากอดีตจนถึงปัจจุบันส่วนใหญ่จะเป็นการพัฒนาความสามารถของการเดินของหุ่นยนต์
เช่น เริ่มต้นจากแรกสุดจะเป็นการพัฒนาให้หุ่นยนต์สามารถเดินหน้าได้ ต่อมาก็เพิ่มความสามารถให้หุ่นยนต์สามารถเดินบนพื้นเอียง พื้นขรุขระ
เดินเลี้ยวซ้ายขวา เดินขึ้นลงบันได ฯลฯ เป็นต้น นอกจากนี้ยังมีการพัฒนาปรับปรุงสมดุลของการเดินแบบสองขาอีกด้วย สมดุลของการเดินสามารถแบ่งได้สองแบบหลัก
คือ การเดินแบบสมดุลสถิต และการเดินแบบสมดุลพลวัต งานในยุคแรกนั้นจะพัฒนาให้เดินได้แบบสมดุลสถิต ต่อมาเป็นสมดุลกึ่งพลวัต
และเป็นสมดุลพลวัต การพัฒนาตัวควบคุมการเดินของหุ่นยนต์ จำเป็นที่จะต้องใช้ความรู้ทางด้านกลศาสตร์ค่อนข้างมาก มีการใช้สมการที่มีความซับซ้อน

Zheng และคณะ (1988) พัฒนาหุ่นยนต์สองขาที่สามารถเดินบนพื้นราบได้ ให้สามารถเดินต่อเนื่องไปบนพื้นเอียงได้ด้วย
พื้นเอียงที่ใช้มีลักษณะเป็นพื้นเอียงขึ้น หุ่นยนต์ที่ใช้ในงานนี้มีข้อต่อสะโพก (hip), ข้อเท้า (ankle) และลำตัว (torso) มีเซนเซอร์วัดแรงกด (force sensor)
ติดตั้งอยู่ที่ปลายเท้าและส้นเท้าแต่ละข้างเพื่อใช้วัดตำแหน่งของน้ำหนักโดยรวม (center of gravity) ของหุ่นยนต์ การเดินของงานวิจัยจะพิจารณาเฉพาะการเดินในแนวหน้าหลัง
โดยมีหลักการคือ การเดินบนพื้นเอียงโดยที่หุ่นยนต์ยังเดินในท่าทางเหมือนกับตอนที่เดินบนพื้นราบจะทำให้น้ำหนักโดยรวมของหุ่นยนต์เลื่อนไปข้างหลัง
ดังนั้นการที่หุ่นยนต์ขยับลำตัวไปด้านหน้าจำทำให้น้ำหนักโดยรวมของหุ่นยนต์กลับมาอยู่ตรงกลางของพื้นที่รับน้ำหนักเหมือนเดิม
ซึ่งจะทำให้หุ่นยนต์มีความสมดุลได้ ดั้งนั้นข้อมูลที่ได้จากหน่วยวัดแรงกดที่เท้าจะถูกนำมาคำนวณตลอดการเดินเพื่อใช้ในการปรับเปลี่ยนมุมการขยับของลำตัว
การเดินบนพื้นราบเป็นแบบสมดุลสถิตและการเดินบนพื้นเอียงก็ยังคงเป็นแบบสมดุลสถิตเช่นกัน

Inaba และคณะ (1995) สร้างหุ่นยนต์เลียนแบบลิง (ape-like biped) ประกอบด้วยสองมือและสองขา มีการเดินแบบสมดุลสถิต
งานวิจัยนี้มีความคิดว่านอกจากการทำให้หุ่นยนต์สองขาเดินได้โดยไม่ล้มแล้ว ควรจะทำหุ่นยนต์ที่สามารถลุกขึ้นเองได้หลังจากที่ล้มแล้วด้วย
ดังนั้นในงานนี้ หุ่นยนต์ถูกพัฒนาให้สามารถเดิน เมื่อล้มแล้วก็สามารถพลิกตัวและลุกขึ้นมาเดินให้ได้

Kun และ Miller (1996) ได้นำโครงข่ายประสาทเทียม มาประยุกต์ใช้ในการปรับเปลี่ยนท่าทางการเดินโดยอัตโนมัติของหุ่นยนต์สองขา
การที่หุ่นยนต์สามารถปรับเปลี่ยนท่าทางได้โดยอัตโนมัตินี้มีประโยชน์ทำให้หุ่นยนต์เดินได้บนพื้นผิวหลากหลายลักษณะมากขึ้น
ในงานนี้พิจารณาท้ังสมดุลในแนวหน้าหลัง (sigittal plane) และแนวซ้ายขวา (frontal plane) และการเดินของหุุ่นยนต์เป็นแบบสมดุลพลวัต
หลักการทำงานประกอบด้วยตัวสร้างท่าทางการเดินหนึ่งตัว และตัวปรับท่าทางการเดินทั้งแนวหน้าหลังและซ้ายขวาอีกหนึ่งตัว
โดยค่าการปรับเปลี่ยนนั้นจะได้มาจาก แรงกดที่เท้า ความยาวการก้าวเท้า ความสูงของการยกเท้า เป็นต้น นอกจากนี้
ในปีถัดมาทั้งสิงได้ใช้หลักการที่ใช้ในงานนี้ไปใช้กับการเดินของหุ่นยนต์อีกตัว (Kun and Miller, 1997)

Hirai และคณะ (1998) พัฒนาหุ่นยนต์ฮิวมานอยด์ ซึ่งตัวหุ่นยนต์มีความคล้ายมนุษย์มาก สามารถเดินได้อย่างราบลื่นคล้านมนุษย์มากที่สุด
เช่น สามารถเดินได้ในพื้นผิวชนิดต่างๆ เดินได้บนพื้นเอียงขึ้นเอียงลง เดินขึ้นลงบันได้ได้ เดินเข็นรถได้ เป็นต้น การเดินในทุกสถานการณ์เป็นการเดินแบบสมดุลพลวัต
หุ่นยนต์สามารถเดินได้ด้วยความเร็วสูงสุด 4.7 กิโลเมตรต่อชั่วโมง หุ่นยนต์ประกอบไปด้วย แขนข้างละ 9 องศาอิสระ ขาข้างละ 6 องศาอิสระ
ที่บริเวณหัวมีกล้องติดตั้งอยู่ 4 ตัว นอกจากนี้ยังมีอุปกรณ์ที่ใช้ในการรักษาสมดุลอื่นๆ อีกได้แก่ IMU ที่ติดตั้งบริเวณลำตัว และ Force sensor ที่ติดที่เท้าทั้งสองข้าง

องค์ประกอบของหุ่นยนต์ทั่วไปจะประกอบไปด้วยกระบวนการตอบสนองต่างๆที่เป็นระบบ
ซึ่งเราสามารถจำแนกกาออกเป็นส่วนหลักๆได้สามส่วนคือ ส่วนการรับรู้ ส่วนการประมวลผล
และส่วนการขับเคลื่อน ทั้งหมดเมื่อนำมารวมเข้าด้วยกันแล้ว เราสามารถที่จะควบคุมการทำงานของหุ่นยนต์ฮิวมานอยด์ได้

\begin{figure}[htbp]
    \centering
    \includegraphics[width=0.45\textwidth]{chapter2/images/humanoid_combine.png}
	\caption{องค์ประกอบหลักของหุ่นยนต์ฮิวมานอยด์}
    \label{fig:humanoid_combine}
\end{figure}

\clearpage
\subsubsection*{การรับรู้ของหุ่นยนต์ฮิวมานอยด์}
การรับรู้ของหุ่นยนต์ฮิวมานอยด์นั้นมีความยากมากกว่าหุ่นยนต์ชนิดอื่นๆเพราะหุ่นยนต์จะมีการเคลื่อนที่ และการเคลื่อนที่นั้นทำให้เซนเซอร์โดนรบกวนได้
ยกตัวอย่างเช่น ภาพที่ได้จากกล้องนั้นอาจจะเบลอได้ถ้าความเร็วของชัตเตอร์ช้าเกินไป หรือว่าภาพเปลี่ยนขณะที่กำลังกดชัตเตอร์
ข้อมูลตำแหน่งของตัวเองก็มีความแน่นอนที่น้อยกว่าหุ่นยนต์เคลื่อนที่ด้วยล้อ เพราะเซนเซอร์ที่วัดตำแหน่งเทียบกับเฟรมโลกไม่มีความเสถียร
หุ่นยนต์ที่เคลื่อนที่ด้วยล้อปกติถ้าติดกล้อง ตัวกล้องจะมีความสูงจากพื้นคงที่ แต่หุ่นยนต์ฮิวมานอยด์ไม่ใช่ โดยหุ่นยนต์ฮิวมานอยด์นั้นจะต้องมีการคำนวณ
forward kinematics จากเท้าที่สัมผัสกับพื้นมายังกล้องเพื่อหาตำแหน่งและการหมุนของกล้อง ส่วนการวัดตำแหน่งของตัวหุ่นยนต์นั้น
โดยทั่วไปแล้วจะใช้เซนเซอร์ inertia measurement unit (IMU) และเซนเซอร์ Encoders สำหรับหาตำแหน่งของข้อต่อต่างๆ
ปกติจะติดเซนเซอร์ IMU ไว้ที่ลำตัวของหุ่นยนต์ใกล้ๆกับ center of mass ของหุ่นยนต์ ส่วน Encoder นั้นจะติดไว้ที่ข้อต่อของหุ่นยนต์

\subsubsection*{การประมวลผลของหุ่นยนต์ฮิวมานอยด์}
ในปัจจุบันนี้หุ่นยนต์ฮิวมานอยด์มีความสามารถในการคำนวณที่สูงมากเมื่อเทียบกับเมื่อก่อน บอร์ดที่เราสามารถเห็นได้โดยทั่วไปเช่น
Raspberry Pi, Odroid, Intel NUCs ซึ่งตัวบอร์ดมีขนาดเล็กจึงทำให้เข้าไปอยู่ในตัวของหุ่นยนต์ได้ แถมบอร์ดพวกนี้ยังมี GPUs
และ CPU หลายคอร์อีกด้วย บางครั้งก็มีคนที่เอาพวกบอร์ดพวกนี้มาทำงานร่วมกันหลายๆตัว ประมวลผลแบบ pararell เพื่อที่จะเพิ่มประสิทธิภาพในการประมวลผล
โดยเชื่อมต่อระหว่างกันผ่าน Ethernet network 

\subsubsection*{การขับเคลื่อนของหุ่นยนต์ฮิวมานอยด์}
หุ่นยนต์ฮิวมานอยด์ส่วนใหญ่จะมีข้อต่ออยู่หลายๆจุด แต่ละข้อต่อจะมีตัวขับเคลื่อน ตัวขับเคลื่อนมีอยู่หลักๆสองแบบคือ
แบบกล้ามเนื่อของมนุษย์ และแบบมอเตอร์ที่ติดตรงที่ข้อต่อเลย ที่นิยมใช้คือแบบมอเตอร์ที่ติดที่ข้อต่อเลย เพราะทำให้ตัวของหุ่นยนต์มีขนาดเล็ก
ใช้พื้นที่น้อย การใช้เส้นเอ็นดึงนั้นจะการจะทำให้ข้อต่อไปยังตำแหน่งที่ต้องการได้ยากกว่า ตัวขับเคลื่อนนั้นต้องการแรงมากน้อยขึ้นอยู่กับ
น้ำหนักของตัวหุ่นยนต์ เพื่อที่จะทำให้หุ่นยนต์นั้นยังยืนได้
